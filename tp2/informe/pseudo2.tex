
\subsection{Algoritmo de tabulado}
\begin{algorithm}
\caption{Construye la tabla de valores precalculados para hallar el camino}
\begin{algorithmic}[1]

\STATE matA $\textcolor{orange}{\leftarrow}$ matrizCuadradaDeCeros\textcolor{SkyBlue}{(}n\textcolor{SkyBlue}{)}
\STATE matB $\textcolor{orange}{\leftarrow}$ matrizCuadradaDeCeros\textcolor{SkyBlue}{(}n\textcolor{SkyBlue}{)}

\COMMENT {Carga los casos base (nodos adyacentes)}
\FOR {$i$ $\textcolor{orange}{\in}$ $0,...,n\textcolor{orange}{-}1$}
    \STATE a $\textcolor{orange}{\leftarrow}$ $i$
    \STATE b $\textcolor{orange}{\leftarrow}$ siguiente\textcolor{SkyBlue}{(}i\textcolor{SkyBlue}{)}
    \IF {estanConectados\textcolor{SkyBlue}{(}a,b\textcolor{SkyBlue}{)}}
        \STATE matA\textcolor{orange}{[}a\textcolor{orange}{]}\textcolor{orange}{[}b\textcolor{orange}{]} $\textcolor{orange}{\leftarrow}$ 1
        \STATE matB\textcolor{orange}{[}a\textcolor{orange}{]}\textcolor{orange}{[}b\textcolor{orange}{]} $\textcolor{orange}{\leftarrow}$ 1
    \ENDIF
\ENDFOR

\COMMENT {Carga los casos recursivos (nodos no adyacentes)}
\FOR {$i$ $\textcolor{orange}{\in}$ $2,...,n\textcolor{orange}{-}1$}
    \FOR {$j$ $\textcolor{orange}{\in}$ $0,...,n\textcolor{orange}{-}1$}
        \STATE a $\textcolor{orange}{\leftarrow}$ j
        \STATE b $\textcolor{orange}{\leftarrow}$ $j\textcolor{orange}{+}i$ mod $n$

        \STATE aux\_a1 $\textcolor{orange}{\leftarrow}$ estanConectados\textcolor{SkyBlue}{(}a,siguiente\textcolor{SkyBlue}{(}a\textcolor{SkyBlue}{)}\textcolor{SkyBlue}{)} $\textcolor{orange}{\wedge}$ matA\textcolor{orange}{[}siguiente\textcolor{SkyBlue}{(}a\textcolor{SkyBlue}{)}\textcolor{orange}{]}\textcolor{orange}{[}b\textcolor{orange}{]}
        \STATE aux\_a2 $\textcolor{orange}{\leftarrow}$ estanConectados\textcolor{SkyBlue}{(}a,b\textcolor{SkyBlue}{)} $\textcolor{orange}{\wedge}$ matB\textcolor{orange}{[}siguiente\textcolor{SkyBlue}{(}a\textcolor{SkyBlue}{)},b\textcolor{orange}{]}
        \STATE aux\_b1 $\textcolor{orange}{\leftarrow}$ estanConectados\textcolor{SkyBlue}{(}b,anterior\textcolor{SkyBlue}{(}b\textcolor{SkyBlue}{)}\textcolor{SkyBlue}{)} $\textcolor{orange}{\wedge}$ matB\textcolor{orange}{[}a\textcolor{orange}{]}\textcolor{orange}{[}anterior\textcolor{SkyBlue}{(}b\textcolor{SkyBlue}{)}\textcolor{orange}{]}
        \STATE aux\_b2 $\textcolor{orange}{\leftarrow}$ estanConectados\textcolor{SkyBlue}{(}a,b\textcolor{SkyBlue}{)} $\textcolor{orange}{\wedge}$ matA\textcolor{orange}{[}a\textcolor{orange}{]}\textcolor{orange}{[}anterior\textcolor{SkyBlue}{(}b\textcolor{SkyBlue}{)}\textcolor{orange}{]}

        \STATE matA\textcolor{orange}{[}a\textcolor{orange}{]}\textcolor{orange}{[}b\textcolor{orange}{]} $\textcolor{orange}{\leftarrow}$ aux\_a1 $\textcolor{orange}{\vee}$ aux\_a2
        \STATE matB\textcolor{orange}{[}a\textcolor{orange}{]}\textcolor{orange}{[}b\textcolor{orange}{]} $\textcolor{orange}{\leftarrow}$ aux\_b1 $\textcolor{orange}{\vee}$ aux\_b2
    \ENDFOR
\ENDFOR
\end{algorithmic}
\end{algorithm}

Se utilizan las funciones auxiliares:
\begin{itemize}
\item \textbf{matrizCuadradaDeCeros(n)}, que construye una matriz cuadrada de $n*n$ inicializada
      completamente con ceros.
\item \textbf{estanConectados(a,b)}, que determina si dos ciudades $a$ y $b$ tienen un acuerdo comercial
      entre s�.
\item \textbf{siguiente(i)} = $i+1$ mod $n$, que corresponde a la ciudad siguiente a $i$ en orden circular
\item \textbf{anterior(i)} = $i-1$ mod $n$, que corresponde a la ciudad anterior a $i$ en orden circular
\end{itemize}

\newpage
\subsection{Algoritmo auxiliar de construcci'on de caminos entre dos nodos}
\begin{algorithm}
\caption{Halla un camino que recorre los nodos entre a y b y termina en a}
\begin{algorithmic}[1]
\REQUIRE que est�n cargados los valores correctos en $matA$ y $matB$
\REQUIRE que exista el camino que se intentar� generar
\IF { $b\textcolor{orange}{-}a$ \textcolor{orange}{==} $1$ $\textcolor{orange}{\vee}$ \textcolor{SkyBlue}{(}$b$ \textcolor{orange}{==} $0$ $\textcolor{orange}{\wedge}$ $a$ \textcolor{orange}{==} $n\textcolor{orange}{-}1$\textcolor{SkyBlue}{)} } %TODO: se puede reemplazar por son ciudades adyacentes?
    \RETURN \textcolor{orange}{[}b,a\textcolor{orange}{]}
\ENDIF
\STATE amas1 $\textcolor{orange}{\leftarrow}$ siguiente\textcolor{SkyBlue}{(}a\textcolor{SkyBlue}{)}
\IF {estanConectados\textcolor{SkyBlue}{(}a,amas1\textcolor{SkyBlue}{)} $\textcolor{orange}{\wedge}$ matA\textcolor{orange}{[}amas1\textcolor{orange}{]}\textcolor{orange}{[}b\textcolor{orange}{]}}
    \RETURN caminoQueTerminaEnA\textcolor{SkyBlue}{(}amas1,b\textcolor{SkyBlue}{)} \textcolor{orange}{+} \textcolor{orange}{[}a\textcolor{orange}{]}
\ENDIF
\IF {estanConectados\textcolor{SkyBlue}{(}a,b\textcolor{SkyBlue}{)} $\textcolor{orange}{\wedge}$ matB\textcolor{orange}{[}amas1\textcolor{orange}{]}\textcolor{orange}{[}b\textcolor{orange}{]}}
    \RETURN caminoQueTerminaEnB\textcolor{SkyBlue}{(}amas1,b\textcolor{SkyBlue}{)} \textcolor{orange}{+} \textcolor{orange}{[}a\textcolor{orange}{]}
\ENDIF
\end{algorithmic}
\end{algorithm}
Se utilizan las funciones auxiliares:
\begin{itemize}
\item \textbf{caminoQueTerminaEnA}, llamada recursiva a este mismo procedimiento.
\item \textbf{caminoQueTerminaEnB}, funci�n an�loga a la aqu� �descripta que genera
      caminos que terminan en B en lugar de hacerlo en A.
\item \textbf{estanConectados(a,b)}, \textbf{siguiente(i)}, \textbf{anterior(i)}: idem 
      que para el algoritmo anterior.
\end{itemize}


\subsection{Algoritmo de b�squeda de caminos en todo el grafo propuesto}
\begin{algorithm}
\caption{Construye un camino apropiado a partir de las tablas precalculadas}
\begin{algorithmic}[1]
\IF { $m \textcolor{orange}{<} n\textcolor{orange}{-}1$ }
%\COMMENT{Si no hay ejes suficientes para que el grafo sea conexo, no hay soluci'on}
\RETURN \textcolor{orange}{[} \textcolor{orange}{]}
\ENDIF
\STATE completarTablas\textcolor{SkyBlue}{(}\textcolor{SkyBlue}{)}
\COMMENT{Completo las tablas $matA$ y $matB$}
\FOR {$i$ $\textcolor{orange}{\in}$ $0,...,n\textcolor{orange}{-}1$}
    \STATE a $\textcolor{orange}{\leftarrow}$ i
    \STATE b $\textcolor{orange}{\leftarrow}$ anterior\textcolor{SkyBlue}{(}i\textcolor{SkyBlue}{)}
    \IF {matA\textcolor{orange}{[}a\textcolor{orange}{]}\textcolor{orange}{[}b\textcolor{orange}{]}}
        \RETURN caminoQueTerminaEnA\textcolor{SkyBlue}{(}a,b\textcolor{SkyBlue}{)}
    \ENDIF
    \IF {matB\textcolor{orange}{[}a\textcolor{orange}{]}\textcolor{orange}{[}b\textcolor{orange}{]}}
        \RETURN caminoQueTerminaEnB\textcolor{SkyBlue}{(}a,b\textcolor{SkyBlue}{)}
    \ENDIF
\ENDFOR
\RETURN \textcolor{orange}{[} \textcolor{orange}{]}
\end{algorithmic}
\end{algorithm}
Se utilizan las funciones auxiliares:
\begin{itemize}
\item \textbf{caminoQueTerminaEnA}, \textbf{caminoQueTerminaEnB}, el procedimiento descripto
      anteriormente y su funci�n an�loga para B como se indic� previamente.
\end{itemize}
