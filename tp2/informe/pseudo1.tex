\begin{algorithm}
\caption{Devuelve la lista de aquellos jugadores, tal que se puede arreglar el torneo}
\begin{algorithmic}[1]
\IF{hay menos partidos arreglados que jugadores $-$ 1}
	\STATE devolver []
\ENDIF
\STATE fuertes $\textcolor{orange}{\leftarrow}$ armarFuertes(grafo)
\COMMENT {Averiguamos en que componente quedo cada nodo}
\FOR {i $\in$ {$1,...,$ Cantidad de componentes fuertemente conexas}}
	\FOR {cada nodo $\in$ $fuertes_i$}
			\STATE dondeQuedo\textcolor{orange}{[}nodo\textcolor{orange}{]} $\textcolor{orange}{\leftarrow}$ i
	\ENDFOR
\ENDFOR
\STATE relacion$\textcolor{orange}{\leftarrow}\textcolor{orange}{[} \textcolor{orange}{]}$
\FOR{cada nodo del grafo}
	\FOR{cada nodo2 que llega al nodo}
			\IF{si el vertice no une elementos de la misma componente}
				\STATE relacion $\textcolor{orange}{+}$ $\textcolor{orange}{[}(dondeQuedo\textcolor{orange}{[}nodo\textcolor{orange}{]},dondeQuedo\textcolor{orange}{[}nodo2\textcolor{orange}{]})\textcolor{orange}{]}$
			\ENDIF
	\ENDFOR
\ENDFOR
\STATE g1 $\textcolor{orange}{\leftarrow}$ Grafo(cantidad de Componentes, relacion)
\COMMENT{Una vez que tengo el grafo reducido, busco cuantos hay con $d_{in} = 0$}
\STATE encontreUno = false
\STATE quien $\textcolor{orange}{\leftarrow}$ $\bot$
\FOR{cada nodo de g1}
	\IF{ $d_{in}(nodo) == 0$ $\wedge$ no encontreUno}
		\STATE quien $\textcolor{orange}{\leftarrow}$ nodo
		\STATE encontreUno $\textcolor{orange}{\leftarrow}$ True
	\ELSIF{ $d_{in}(nodo) == 0$ $\wedge$ encontreUno}
		\STATE devolver $\textcolor{orange}{[}\textcolor{orange}{]}$
	\ENDIF
\ENDFOR
\STATE $ordenar(fuertes\textcolor{orange}{[}quien\textcolor{orange}{]})$ \COMMENT{lo hago con bucket sort, ya que se que estan entre 0 y cantidad de nodos del grafo original}
\STATE devolver $fuertes\textcolor{orange}{[}quien\textcolor{orange}{]}$	
\end{algorithmic}
\end{algorithm}

\begin{algorithm}
\caption{Devuelve la lista de las componentes fuertemente conexas mediante algoritmo de Kosaraju}
\begin{algorithmic}[1]
\STATE valor = {0...0}
\STATE dar valor a todos los nodos segun las llamadas recursivas al hacer dsf
\STATE $visitado = {0...0}$
\STATE g $\textcolor{orange}{\leftarrow}$ invertirGrafo(grafo)
\STATE $CompFuertes \textcolor{orange}{\leftarrow} \textcolor{orange}{[}\textcolor{orange}{]}$
\FOR{cada nodo con mayor valor y sin visitar}
		\STATE fuerte $\textcolor{orange}{\leftarrow}$ \textcolor{orange}{[}\textcolor{orange}{]}
		\STATE Realizar dsf desde el nodo, los elementos visitados son una componente fuertemente conexa
		\STATE fuertes $\textcolor{orange}{\leftarrow}$ fuertes \textcolor{orange}{+} \textcolor{orange}{[}fuerte\textcolor{orange}{]}
\ENDFOR
\STATE devolver fuertes
\end{algorithmic}
\end{algorithm}

%\begin{algorithm}
%\caption{numera los nodos mediante dsf}
%\begin{algorithmic}\textcolor{orange}{[}1\textcolor{orange}{]}
%\STATE $visitado \textcolor{orange}{\leftarrow} {0...0}$
%\FOR{cada nodo del grafo}
%	\IF{no lo visite}
%		\STATE dsf(grafo, nodo,valor) \COMMENT{dsf que numera}		
%	\ENDIF
%	\ENDFOR
%	
%\end{algorithmic}
%\end{algorithm}
%
%\begin{algorithm}
%\caption{Realiz un dsf numerando a cada nodo segun el orden de su llamada, guarda que valor tiene cada nodo}
%\begin{algorithmic}\textcolor{orange}{[}1\textcolor{orange}{]}
%\STATE visitado\textcolor{orange}{[}nodo\textcolor{orange}{]} $\textcolor{orange}{\leftarrow}$ 1
%\FOR{Cada vertice relacionado con el nodo}
%	\IF{no lo visite}
%		\STATE bfp(grafo,vertice}
%	\ENDIF
%\ENDFOR
%\STATE Valor\textcolor{orange}{[}valorActual\textcolor{orange}{]} $\textcolor{orange}{\leftarrow}$ nodo
%\STATE valorActual $\textcolor{orange}{+}\textcolor{orange}{+}$
%
