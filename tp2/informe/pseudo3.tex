\begin{algorithm}
\caption{agrega una palabra s al conjunto}
\begin{algorithmic}[1]
\STATE bajar por las ramas hasta no encontrar mas prefijo en el trie
\STATE palabraArmada \textcolor{orange}{$\leftarrow$} palabra armada durante el recorrido
\STATE acutal \textcolor{orange}{$\leftarrow$} nodo hasta donde baje
\STATE ejeActual \textcolor{orange}{$\leftarrow$} eje que apunta al nodo actual
\IF{si la palabra ya esta definida}
	\STATE no hacer nada
\ELSE
	\STATE elminar prefijos comunes de s y de palabraArmada
\IF{palabrArmada $\neq$ $""$} %\COMMENT{}
		\STATE \COMMENT{hay que partir un nodo} 
		\STATE existia \textcolor{orange}{$\leftarrow$} nodoActual.existe
		\STATE ejeActual.cadena \textcolor{orange}{$\leftarrow$} Ejactual.cadena \textcolor{orange}{-} palabraArmada  \COMMENT{dejar en el eje actual la parte que concidia con s}
		\STATE padreActual \textcolor{orange}{$\leftarrow$} nuevoNodo 
		\STATE crear eje (j) entre padreActual y nodoActual
		\STATE j.cadena \textcolor{orange}{$\leftarrow$} palabraArmada
		\STATE apuntar ejeActual a nuevoPadre 
		\IF{existia}
			\STATE setear existe del nodoActual
		\ELSE
			\STATE seterar en no existe al nodoActual
		\ENDIF
\ENDIF
\ENDIF
	\IF{ si s $\neq$ $""$}
		\STATE agregar un eje con s al nodo actual
		\STATE poner un nodo al final de ese eje, con el existe en true
	\ELSE
		\STATE setear el existe del nodo actual en true
	\ENDIF
\STATE palabras definidas \textcolor{orange}{++}
\end{algorithmic}
\end{algorithm}

\begin{algorithm}
\caption{saca una palabra s del conjunto}
\begin{algorithmic}[1]
\STATE bajar por las ramas hasta no encontrar mas prefijo
\STATE ejeAnterior \textcolor{orange}{$\leftarrow$} ultimo eje que pase
\STATE actual \textcolor{orange}{$\leftarrow$} nodo en el que quede parado
\IF{la palabra esta definida}
	\IF{ el nodo actual es una hoja}
		\STATE elminar nodo y eje
		\IF{el padre de la hoja ahora solo tiene un hijo, y el no esta definido} \STATE \COMMENT{si el nodo esta definido, es porque el conjunto no era libre de prefijos}
		\STATE agregar al eje del padre, la cadena del eje del hijo
		\STATE linkear al padre con los hijos de su hijo
		\STATE borrar al hijo
		\STATE setear al padre como definido
		\ENDIF
	\ELSE
		\IF{el nodo a borrar tiene solo un hijo}
		\STATE \COMMENT{este caso no puede ocurrir en un conjunto libre de prefijos}
		\STATE combinar las cadenas del eje desde el padre de actual a actual y el eje desde actual a su hijo 
		\STATE borrar actual
		\ELSE
				\STATE \COMMENT{tiene mas de un hijo}
				\STATE setear actual como no definido
		\ENDIF
	\ENDIF
	\STATE palabras definidas \textcolor{orange}{--}
\ENDIF
\end{algorithmic}
\end{algorithm}		

\begin{algorithm}
\caption{determina si una palabras esta en el conjunto}
\begin{algorithmic}[1]
			\STATE bajar por las ramas hasta que no quede mas prefijo
			\STATE palabraArmada \textcolor{orange}{$\leftarrow$} palabra armada durante el recorrido
			\STATE devolver palabraArmada \textcolor{orange}{==} s \textcolor{orange}{$\wedge$} el nodo existe
\end{algorithmic}
\end{algorithm}		

\begin{algorithm}
\caption{baja por las ramas segun una cadena s, ademas va armando la palabra que se forma durante el recorrido }
\begin{algorithmic}[1]
\STATE cadenaArmada \textcolor{orange}{$\leftarrow$}  $""$
\STATE ejeAnterior \textcolor{orange}{$\leftarrow$}  \textcolor{Tan}{NULL}
\STATE ejeActual\textcolor{orange}{$\leftarrow$}  \textcolor{Tan}{NULL}
\STATE nodoActual \textcolor{orange}{$\leftarrow$} raiz
\STATE eje \textcolor{orange}{$\leftarrow$} eje de nodo actual que comienza con la primer letra de s
\STATE puedoBajar \textcolor{orange}{$\leftarrow$} (eje \textcolor{orange}{$\neq$} \textcolor{Tan}{NULL});
\WHILE{ eje \textcolor{orange}{$\neq$} \textcolor{Tan}{NULL} \textcolor{orange}{$\wedge$} puedoBajar }
	\STATE ejeAnterior \textcolor{orange}{$\leftarrow$} ejeActual
	\STATE ejeActual \textcolor{orange}{$\leftarrow$} eje
	\STATE nodoActual \textcolor{orange}{$\leftarrow$} nodo apuntado por el eje
	\STATE palabraArmada \textcolor{orange}{$\leftarrow$} palabraArmada \textcolor{orange}{++} cadena del eje
	\STATE sacar a s la parte del eje que le es prefijo
	\STATE puedoBajar \textcolor{orange}{$\leftarrow$} el eje entero era prefijo de s
	\STATE eje \textcolor{orange}{$\leftarrow$} eje del nodo actual que empieza con la primer letra de s
\ENDWHILE
\STATE devolver ejeAnterior
\end{algorithmic}
\end{algorithm}	



