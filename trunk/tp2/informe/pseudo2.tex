
\subsection{Algoritmo de tabulado}
\begin{algorithm}
\caption{Construye la tabla de valores precalculados para hallar el camino}
\begin{algorithmic}[1]

\STATE matA $\textcolor{orange}{\leftarrow}$ matrizCuadradaDeCeros(n)
\STATE matB $\textcolor{orange}{\leftarrow}$ matrizCuadradaDeCeros(n)

\COMMENT {Carga los casos base (nodos adyacentes)}
\FOR {$i$ $\in$ $0,...,n-1$}
    \STATE a $\textcolor{orange}{\leftarrow}$ $i$
    \STATE b $\textcolor{orange}{\leftarrow}$ $i+1$ mod $n$
    \IF {estanConectados(a,b)}
        \STATE matA[a][b] $\textcolor{orange}{\leftarrow}$ 1
        \STATE matB[a][b] $\textcolor{orange}{\leftarrow}$ 1
    \ENDIF
\ENDFOR

\COMMENT {Carga los casos recursivos (nodos no adyacentes)}
\FOR {$i$ $\in$ $2,...,n-1$}
    \FOR {$j$ $\in$ $0,...,n-1$}
        \STATE a $\textcolor{orange}{\leftarrow}$ j
        \STATE b $\textcolor{orange}{\leftarrow}$ $j+i$ mod $n$
        \STATE amas1 $\textcolor{orange}{\leftarrow}$ $a+1$ mod $n$
        \STATE bmenos1 $\textcolor{orange}{\leftarrow}$ $b-1$ mod $n$

        \STATE aux\_a1 $\textcolor{orange}{\leftarrow}$ estanConectados(a,amas1) y matA[amas1][b]
        \STATE aux\_a2 $\textcolor{orange}{\leftarrow}$ estanConectados(a,b) y matB[amas1,b]
        \STATE aux\_b1 $\textcolor{orange}{\leftarrow}$ estanConectados(b,bmenos1) y matB[a][bmenos1]
        \STATE aux\_b2 $\textcolor{orange}{\leftarrow}$ estanConectados(a,b) y matA[a][bmenos1]

        \STATE matA[a][b] $\textcolor{orange}{\leftarrow}$ aux\_a1 o aux\_a2
        \STATE matB[a][b] $\textcolor{orange}{\leftarrow}$ aux\_b1 o aux\_b2
    \ENDFOR
\ENDFOR
\end{algorithmic}
\end{algorithm}

Se utilizan las funciones auxiliares:
\begin{itemize}
\item \textbf{matrizCuadradaDeCeros(n)}, que construye una matriz cuadrada de $n*n$ inicializada
      completamente con ceros.
\item \textbf{estanConectados(a,b)}, que determina si dos ciudades $a$ y $b$ tienen un acuerdo comercial
      entre s�.
\end{itemize}

\newpage
\subsection{Algoritmo auxiliar de construcci'on de caminos entre dos nodos}
\begin{algorithm}
\caption{Halla un camino que recorre los nodos entre a y b y termina en A}
\begin{algorithmic}[1]
\REQUIRE que est�n cargados los valores correctos en $matA$ y $matB$
\REQUIRE que exista el camino que se intentar� generar
\IF { $b-a$ == $1$ o ($b$ == $0$ y $a$ == $n-1$) }
    \RETURN [b,a]
\ENDIF
\STATE amas1 $\textcolor{orange}{\leftarrow}$ $a+1$ mod $n$
\IF {estanConectados(a,amas1) y matA[amas1][b]}
    \RETURN caminoQueTerminaEnA(amas1,b) + [a]
\ENDIF
\IF {estanConectados(a,b) y matB[amas1][b]}
    \RETURN caminoQueTerminaEnB(amas1,b) + [a]
\ENDIF
\end{algorithmic}
\end{algorithm}
Se utilizan las funciones auxiliares:
\begin{itemize}
\item \textbf{caminoQueTerminaEnA}, llamada recursiva a este mismo procedimiento.
\item \textbf{caminoQueTerminaEnB}, funci�n an�loga a la aqu� �descripta que genera
      caminos que terminan en B en lugar de hacerlo en A.
\item \textbf{estanConectados(a,b)}, idem que para el algoritmo anterior.
\end{itemize}


\subsection{Algoritmo de b�squeda de caminos en todo el grafo propuesto}
\begin{algorithm}
\caption{Construye un camino apropiado a partir de las tablas precalculadas}
\begin{algorithmic}[1]
\IF { $m < n-1$ }
\RETURN [ ]
\ENDIF
\STATE completarTablas()
\COMMENT{Completo las tablas $matA$ y $matB$}
\FOR {$i$ $\in$ $0,...,n-1$}
    \STATE a $\textcolor{orange}{\leftarrow}$ i
    \STATE b $\textcolor{orange}{\leftarrow}$ $i-1$ mod $n$
    \IF {matA[a][b]}
        \RETURN caminoQueTerminaEnA(a,b)
    \ENDIF
    \IF {matB[a][b]}
        \RETURN caminoQueTerminaEnB(a,b)
    \ENDIF
\ENDFOR
\RETURN [ ]
\end{algorithmic}
\end{algorithm}
Se utilizan las funciones auxiliares:
\begin{itemize}
\item \textbf{caminoQueTerminaEnA}, \textbf{caminoQueTerminaEnB}, el procedimiento descripto
      anteriormente y su funci�n an�loga para B como se indic� previamente.
\end{itemize}
