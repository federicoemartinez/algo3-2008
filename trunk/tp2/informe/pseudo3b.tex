\begin{algorithm}[H]
\caption{saca una palabra s del conjunto}
\begin{algorithmic}[1]
\PARAMS{ la palabra s a sacar}
\REQUIRE{que la palabra a sacar halla sido agregada}
\STATE bajar por las ramas hasta no encontrar mas prefijo
\STATE ejeAnterior \textcolor{orange}{$\leftarrow$} ultimo eje que pase
\STATE actual \textcolor{orange}{$\leftarrow$} nodo en el que quede parado
	\IF{ el nodo actual es una hoja}
		\STATE elminar nodo y eje
		\IF{el padre de la hoja ahora solo tiene un hijo, y el no esta definido} \STATE \COMMENT{si el nodo esta definido, es porque el conjunto no era libre de prefijos}
		\STATE agregar al eje del padre, la cadena del eje del hijo
		\STATE linkear al padre con los hijos de su hijo
		\STATE borrar al hijo
		\STATE setear al padre como definido
		\ENDIF
	\ELSE
		\IF{el nodo a borrar tiene solo un hijo}
		\STATE \COMMENT{este caso no puede ocurrir en un conjunto libre de prefijos}
		\STATE combinar las cadenas del eje desde el padre de actual a actual y el eje desde actual a su hijo 
		\STATE borrar actual
		\ELSE
				\STATE \COMMENT{tiene mas de un hijo}
				\STATE setear actual como no definido
		\ENDIF
	\ENDIF
	\STATE palabras definidas \textcolor{orange}{--}
\end{algorithmic}
\end{algorithm}		