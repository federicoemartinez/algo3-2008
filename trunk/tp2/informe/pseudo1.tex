\begin{algorithm}[H]
\caption{Devuelve la lista de aquellos jugadores, tal que se puede arreglar el torneo}
\label{alg:algoritmo1}
\begin{algorithmic}[1]
\PARAMS {Grafo con jugadores como nodos, y partidos arreglados como aristas direccionadas}
\IF{hay menos partidos arreglados que jugadores \textcolor{orange}{$-$} 1}
	\RETURN \textcolor{orange}{[ ]}
\ENDIF
\STATE \textcolor{MidnightBlue}{DFS} numerando todos los nodos por fin de llamada recursiva
\STATE hacer \textcolor{MidnightBlue}{DFS} solo para el nodo de mayor numeraci�n
\STATE aCuantosLlega $\textcolor{orange}{\leftarrow}$ cantidad de nodos tocados en el \textcolor{MidnightBlue}{DFS} anterior
\IF{aCuantosLlega $\textcolor{orange}{==}$ todos los nodos del grafo}
	\STATE hacer \textcolor{MidnightBlue}{DFS} desde el nodo pero con las aristas de salida
	\STATE ganadores $\textcolor{orange}{\leftarrow}$ todos los nodos tocados por el \textcolor{MidnightBlue}{DFS} anterior
	\STATE ordenar\textcolor{SkyBlue}{(}ganadores\textcolor{SkyBlue}{)} \COMMENT{ lo hago con bucketSort porque los ganadores estan entre 1 y cantidad de nodos}
\ELSE
	\RETURN \textcolor{orange}{[ ]}
\ENDIF
\end{algorithmic}
\end{algorithm}
	
\begin{algorithm}[H]
\caption{numera nodos, guarda quien el que tiene el valor mas alto cantidad de nodos. Esta funci�n debe llamarse por primera vez con el contador inicializado en cero y todos los visitados en \textit{false}. Cada vez que termina, si no se visitaron todos debe ser llamada de nuevo, manteniendo el contador y los visitados.}
\label{alg:algoritmo1b}
\begin{algorithmic}[1]
\PARAMS{un grafo, un nodo del grafo, un contador que se modifica, un marcador de nodos visitados}
\PARAMS{una variable que se modifica y guarda el nodo de max valor}
\STATE marcar al nodo como visitado
\FOR{cada nodoAdyacente tal que nodo \textcolor{orange}{$\leadsto$} nodoAdyacente}
	\IF{ no lo visite}
		\STATE dfs\textcolor{SkyBlue}{(}grafo,nodoAdyacente,contador,visitados,masValioso\textcolor{SkyBlue}{)}
	\ENDIF
\ENDFOR
\IF{a este nodo le toca el valor mas alto}
	\STATE valioso $\textcolor{orange}{\leftarrow}$ nodo
\ENDIF
\STATE incrementar el contador
\end{algorithmic}
\end{algorithm}

\begin{algorithm}[H]
\caption{Cuenta a cuantos nodos visite, partiendo desde un nodo dado. La usamos para determinar si partiendo del de mayor n�mero (c�lculado por el algoritmo anterior) podemos llegar a todos}
\label{alg:algoritmo1c}
\begin{algorithmic}[1]
\PARAMS{un grafo, un nodo del grafo, un marcador de nodos visitados}
\STATE marcar al nodo como visitado
\STATE cuantos \textcolor{orange}{$\leftarrow$} 0
\FOR{cada nodoAdyacente tal que nodo \textcolor{orange}{$\leadsto$} nodoAdyacente}
	\IF{ no lo visite}
		\STATE cuantos \textcolor{orange}{$\leftarrow$} cuantos + dfs\textcolor{SkyBlue}{(}grafo,nodoAdyacente,visitados\textcolor{SkyBlue}{)}
	\ENDIF
\ENDFOR
\RETURN cuantos
\end{algorithmic}
\end{algorithm}

\begin{algorithm}[H]
\caption{Lista los nodos visitados haciendo dfs desde un nodo dado, pero usando las aristas que llegan a el. Si se usa partiendo desde el nodo de mayor numeraci�n, se obtiene su componente fuertemente conexa}
\label{alg:algoritmo1b}
\begin{algorithmic}[1]
\PARAMS{un grafo, un nodo del grafo, un contador que se modifica, una lista donde ir guardando los nodos}
\STATE marcar al nodo como visitado
\STATE \COMMENT{notar que a diferencia con los DFS anteriores, usamos las aristas de llegada}
\FOR{cada nodoAdyacente tal que nodoAdyacente \textcolor{orange}{$\leadsto$} nodo} 
	\IF{ no lo visite}
		\STATE agergar nodo a la lista
		\STATE dfs\textcolor{SkyBlue}{(}grafo,nodoAdyacente,visitados\textcolor{SkyBlue}{)}
	\ENDIF
\ENDFOR
\end{algorithmic}
\end{algorithm}