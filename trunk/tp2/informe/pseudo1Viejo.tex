\begin{algorithm}
\caption{Devuelve la lista de aquellos jugadores, tal que se puede arreglar el torneo}
\label{alg:algoritmo1}
\begin{algorithmic}[1]
\PARAMS {Grafo con jugadores como nodos, y partidos arreglados como aristas direccionadas}
\IF{hay menos partidos arreglados que jugadores \textcolor{orange}{$-$} 1}
	\RETURN \textcolor{orange}{[ ]}
\ENDIF
\STATE fuertes $\textcolor{orange}{\leftarrow}$ armarFuertes\textcolor{SkyBlue}{(}grafo\textcolor{SkyBlue}{)}
\COMMENT {Averiguamos en que componente quedo cada nodo}
\FOR {i \textcolor{orange}{$\in$} {$1,...,$ Cantidad de componentes fuertemente conexas}}
	\FOR {cada nodo \textcolor{orange}{$\in$} $fuertes_i$}
			\STATE dondeQuedo\textcolor{orange}{[}nodo\textcolor{orange}{]} $\textcolor{orange}{\leftarrow}$ i
	\ENDFOR
\ENDFOR
\STATE relacion$\textcolor{orange}{\leftarrow}\textcolor{orange}{[} \textcolor{orange}{ ]}$
\FOR{cada nodo del grafo}
	\FOR{cada nodo2 al que llega nodo}
			\IF{si el vertice no une elementos de la misma componente}
				\STATE relacion $\textcolor{orange}{+}$ $\textcolor{orange}{[ }\textcolor{SkyBlue}{(}dondeQuedo\textcolor{orange}{[}nodo\textcolor{orange}{]},dondeQuedo\textcolor{orange}{[}nodo2\textcolor{orange}{]}\textcolor{SkyBlue}{)}\textcolor{orange}{ ]}$
			\ENDIF
	\ENDFOR
\ENDFOR
\STATE g1 $\textcolor{orange}{\leftarrow}$ Grafo\textcolor{SkyBlue}{(}cantidad de Componentes, relacion\textcolor{SkyBlue}{)}
\COMMENT{Una vez que tengo el grafo reducido, busco cuantos hay con $d_{in} = 0$}
\STATE encontreUno $\textcolor{orange}{\leftarrow}$ false
\STATE quien $\textcolor{orange}{\leftarrow}$ \textcolor{orange}{$\bot$}
\FOR{cada nodo de g1}
	\IF{ $d_{in}\textcolor{SkyBlue}{(}nodo\textcolor{SkyBlue}{)} \textcolor{orange}{==} 0$ \textcolor{orange}{$\wedge$} no encontreUno}
		\STATE quien $\textcolor{orange}{\leftarrow}$ nodo
		\STATE encontreUno $\textcolor{orange}{\leftarrow}$ True
	\ELSIF{ $d_{in}\textcolor{SkyBlue}{(}nodo\textcolor{SkyBlue}{)} \textcolor{orange}{==} 0$ \textcolor{orange}{$\wedge$} encontreUno}
		\RETURN  $\textcolor{orange}{[ }\textcolor{orange}{ ]}$
	\ENDIF
\ENDFOR
\STATE $ordenar\textcolor{SkyBlue}{(}fuertes\textcolor{orange}{[}quien\textcolor{orange}{]}\textcolor{SkyBlue}{)}$ \COMMENT{lo hago con bucket sort, ya que se que estan entre 1 y cantidad de nodos del grafo original}
\RETURN $fuertes\textcolor{orange}{[}quien\textcolor{orange}{]}$	
\end{algorithmic}
\end{algorithm}

\begin{algorithm}
\caption{Devuelve la lista de las componentes fuertemente conexas mediante algoritmo de Kosaraju}
\label{alg:algoritmo2}
\begin{algorithmic}[1]
\PARAMS{Grafo con jugadores como nodos, y partidos arreglados como aristas direccionadas}
\STATE valor $\textcolor{orange}{\leftarrow}$ {0...0}
\STATE dar valor a todos los nodos segun las llamadas recursivas al hacer dsf
\STATE en valor\textcolor{orange}{[}i\textcolor{orange}{]} colocar al nodo que tiene valor i
\STATE $visitado \textcolor{orange}{\leftarrow} {0...0}$
\STATE g $\textcolor{orange}{\leftarrow}$ invertirGrafo\textcolor{SkyBlue}{(}grafo\textcolor{SkyBlue}{)}
\STATE $CompFuertes \textcolor{orange}{\leftarrow} \textcolor{orange}{[}\textcolor{orange}{]}$
\FOR{cada nodo con mayor valor y sin visitar}
		\STATE fuerte $\textcolor{orange}{\leftarrow}$ \textcolor{orange}{[}\textcolor{orange}{]}
		\STATE Realizar dsf desde el nodo, los elementos visitados son una componente fuertemente conexa
		\STATE $CompFuertes$ $\textcolor{orange}{\leftarrow}$ $CompFuertes$ \textcolor{orange}{+} \textcolor{orange}{[}fuerte\textcolor{orange}{]}
\ENDFOR
\RETURN fuertes
\end{algorithmic}
\end{algorithm}

\begin{algorithm}
\caption{Genera el grafo inverso de un grafo dado}
\label{alg:algoritmo3}
\begin{algorithmic}
\PARAMS{Grafo que se desea invertir}
\STATE relaciones $\textcolor{orange}{\leftarrow}$ [ ]
\FOR{ cada nodo a del grafo}
	\FOR{ cada otro nodo b del grafo, tal que a$\leadsto$b}
		\STATE agregar a relaciones (b,a)
	\ENDFOR
\ENDFOR
\RETURN Grafo(cantidad de nodos del grafo, relaciones)	
\end{algorithmic}
\end{algorithm}

No se incluyen los algoritmos que usan dsf, ya que hacen exactamente dsf y, o bien numeran los nodos o bien los agregan a una lista. Creemos que no es necesario el pseudocodigo de dichos algoritmos.
			
%\begin{algorithm}
%\caption{numera los nodos mediante dsf}
%\begin{algorithmic}\textcolor{orange}{[}1\textcolor{orange}{]}
%\STATE $visitado \textcolor{orange}{\leftarrow} {0...0}$
%\FOR{cada nodo del grafo}
%	\IF{no lo visite}
%		\STATE dsf(grafo, nodo,valor) \COMMENT{dsf que numera}		
%	\ENDIF
%	\ENDFOR
%	
%\end{algorithmic}
%\end{algorithm}
%
%\begin{algorithm}
%\caption{Realiz un dsf numerando a cada nodo segun el orden de su llamada, guarda que valor tiene cada nodo}
%\begin{algorithmic}\textcolor{orange}{[}1\textcolor{orange}{]}
%\STATE visitado\textcolor{orange}{[}nodo\textcolor{orange}{]} $\textcolor{orange}{\leftarrow}$ 1
%\FOR{Cada vertice relacionado con el nodo}
%	\IF{no lo visite}
%		\STATE bfp(grafo,vertice}
%	\ENDIF
%\ENDFOR
%\STATE Valor\textcolor{orange}{[}valorActual\textcolor{orange}{]} $\textcolor{orange}{\leftarrow}$ nodo
%\STATE valorActual $\textcolor{orange}{+}\textcolor{orange}{+}$
%
