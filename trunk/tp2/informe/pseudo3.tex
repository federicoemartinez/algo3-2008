\begin{algorithm}
\caption{baja por las ramas segun una cadena s, ademas va armando la palabra que se forma durante el recorrido }
\begin{algorithmic}[1]
\PARAMS{una cadena s y otra cadena, cadenaArmada que se va a modificar}
\PARAMS{Tambien un apuntador a nodo que va a puntar al nodo donde terminamos}
\PARAMS{ y otro puntero a eje que va a quedar apuntando al ultimo eje}
\STATE cadenaArmada \textcolor{orange}{$\leftarrow$}  `` ''
\STATE ejeAnterior \textcolor{orange}{$\leftarrow$}  \textcolor{orange}{$\bot$}
\STATE ejeActual\textcolor{orange}{$\leftarrow$}  \textcolor{orange}{$\bot$}
\STATE nodoActual \textcolor{orange}{$\leftarrow$} raiz
\STATE eje \textcolor{orange}{$\leftarrow$} eje de nodo actual que comienza con la primer letra de s
\STATE puedoBajar \textcolor{orange}{$\leftarrow$} (existe dicho eje);
\WHILE{ exista el eje \textcolor{orange}{$\wedge$} puedoBajar }
	\STATE guardamos en eje anterior el ejeActual
	\STATE guardamos en ejeActual el eje que estamos mirando ahora 
	\STATE nodoActual \textcolor{orange}{$\leftarrow$} nodo apuntado por el eje
	\STATE concatenar( palabraArmada, cadena del eje)
	\STATE sacar a s la parte del eje que le es prefijo
	\STATE puedoBajar \textcolor{orange}{$\leftarrow$} el eje entero era prefijo de s
	\STATE eje \textcolor{orange}{$\leftarrow$} eje del nodo actual que empieza con la primer letra de s
\ENDWHILE
\RETURN ejeAnterior
\label{alg:bajar}
\end{algorithmic}
\end{algorithm}	

\begin{algorithm}
\caption{agrega una palabra s al conjunto}
\begin{algorithmic}[1]
\PARAMS{ la palabra s a agregar}
\STATE bajar por las ramas hasta no encontrar mas prefijo en el trie
\STATE palabraArmada \textcolor{orange}{$\leftarrow$} palabra armada durante el recorrido
\STATE acutal \textcolor{orange}{$\leftarrow$} nodo hasta donde baje
\STATE ejeActual \textcolor{orange}{$\leftarrow$} eje que apunta al nodo actual
\IF{ la palabra ya esta definida}
	\STATE no hacer nada
\ELSE
	\STATE elminar prefijos comunes de s y de palabraArmada
\IF{palabrArmada \textcolor{orange}{$\neq$} `` ''} %\COMMENT{}
		\STATE \COMMENT{hay que partir un nodo} 
		\STATE existia \textcolor{orange}{$\leftarrow$} nodoActual.existe
		\STATE cadena del eje acutal \textcolor{orange}{$\leftarrow$} cadena del eje actual \textcolor{orange}{-} palabraArmada  \COMMENT{dejar en el eje actual la parte que concidia con s}
		\STATE padreActual \textcolor{orange}{$\leftarrow$} nuevoNodo 
		\STATE crear eje (j) entre padreActual y nodoActual
		\STATE cadena de j \textcolor{orange}{$\leftarrow$} palabraArmada
		\STATE apuntar ejeActual a nuevoPadre 
		\IF{existia}
			\STATE poner que existe el nodoActual
		\ELSE
			\STATE poner que no existe el nodoActual
		\ENDIF
\ENDIF
\ENDIF
	\IF{ si s \textcolor{orange}{$\neq$} `` ''}
		\STATE agregar un eje con s al nodo actual
		\STATE poner un nodo al final de ese eje, con el existe en true
	\ELSE
		\STATE setear el existe del nodo actual en true
	\ENDIF
\STATE una palabra mas definida
\end{algorithmic}
\end{algorithm}

\begin{algorithm}
\caption{determina si una palabras esta en el conjunto}
\begin{algorithmic}[1]
\PARAMS{ la palabra s a buscar}
		\STATE nodoActual \textcolor{orange}{$\leftarrow$} raiz
		\WHILE{no termine de recorrer s, y no llegue a una hoja}
			\STATE ejeActual \textcolor{orange}{$\leftarrow$}	eje del nodo actual que empieza con la letra de s que estoy mirando
			\IF{no existe ese eje}
				\RETURN false
				\ENDIF
			\WHILE{no terminamos de ver toda s \textcolor{orange}{$\wedge$} no termine de ver toda la cadena del eje}
			\IF{difieren en un caracter}
				\RETURN false
				\ENDIF
				\STATE mirar el proximo caracter de s y de la cadena del eje	
		\ENDWHILE
		\IF{terminamos de ver s pero me quedaron letras en el eje}
		\STATE\COMMENT{quiere decir que s muere en la mitad de un eje}
		\RETURN false
		\ENDIF
		\STATE nodoActual \textcolor{orange}{$\leftarrow$} nodo apuntado por el eje actual
		\ENDWHILE
		 \RETURN miramos toda s \textcolor{orange}{$\wedge$} el nodo donde quedamos	parados existe	
\end{algorithmic}
\end{algorithm}		





